\chapter{Considerações finais}\label{conclusoes}

Este  material  apresentou as  principais  funcionalidades  do framework  Grails
através da  descrição dos aspectos relacionados ao  desenvolvimento da aplicação
{\bf ControleBancario}.

\vspace{0.2cm}
\noindent Para finalizar esse material, esse capítulo apresenta o {\it overview}
de mais algumas funcionalidades presentes  em Grails que não foram abordadas nos
capítulos anteriores.

\section{Automação de testes}

\vspace{0.5cm}

Essa seção apresenta algumas características de Grails que facilitam a automação
de  testes. Relembrando:  testes automatizados  te dão  segurança no  momento da
manutenção. No entanto, antes do início da discussão sobre tais características,
é fundamental relembrar a diferença entre testes unitários e de integração. 

\vspace{0.5cm}
\noindent {\bf Testes unitários} levam em consideração a unidade a ser verificada
isoladamente. Não  há conexões  com bancos  de dados ou  qualquer outro  tipo de
componente: a unidade  deve ser vista como um elemento  isolado (não se comunica
com ninguém).

\vspace{0.5cm}
\noindent {\bf Testes de integração}, por sua vez, levam em consideração, como o
próprio  nome já  diz a  integração  da unidade  a ser  testada com  componentes
externos,  como por  exemplo, bancos  de dados  ou outros  serviços  de natureza
diversa.  Testes de  integração  são, portanto,  mais  caros do  ponto de  vista
computacional, visto que é necessário  iniciar a aplicação para que estes possam
ser executados. 

Todos  os testes  se encontram  no  diretório {\bf  test/unit (ou  integration)}
presente na raiz do projeto Grails. Toda vez que é criada uma classe de domínio,
controlador  (e  outros  artefatos),  testes automaticamente  são  incluídos  no
diretório {\bf  test/unit}.  Da mesma forma  que o GORM é  fortemente baseado no
framework {\it hibernate}, o arcabouço de testes presente em Grails é fortemente
baseado no {\it Junit}\footnote{\url{http://junit.org/}}.  

Há três maneiras de se criar estes testes: (1) O Grails os cria automaticamente;
(2) A classe de teste é criada manualmente pelos desenvolvedores; e (3) A classe
de teste é criada usando o comando {\bf grails create-unit-test}.
\index{Comando!grails create-unit-test}

Assim  como diversos aspectos  do Grails,  aqui é  necessário ater-se  à algumas
convenções.   Toda classe  de teste  possui o  sufixo {\bf  Tests} em  seu nome.
Sendo  assim, os  testes  unitários para  a  classe de  domínio {\bf  Usuario}},
por exemplo, ficariam em {\bf test/unit/UsuarioTests.groovy}.

O comando {\bf grails  create-unit-test} ou {\bf grails create-integration-test}
deve receber  o nome do  teste unitário  ou de integração  a ser gerado.   Não é
necessário  incluir  o  {\bf  Tests}  no  final  do  arquivo,  Grails  o  inclui
automaticamente.  
\index{Comando!grails create-integration-test}

\vspace{0.5cm}
\noindent {\bf Testando  classes de domínio.} Ao lidar  com linguagens dinâmicas
como Groovy  é frequente a  necessidade de lidar  com o seguinte  problema: como
testar uma classe que contém métodos e atributos que só serão injetados em tempo
de execução?  Funções  como {\bf save()}, {\bf validate()}  ou {\bf constraints}
só funcionam após injetados pelo framework.

Uma possibilidade é  escrever testes de integração. O problema  é que leva tempo
até  a  aplicação  ser  iniciada  --  o que  provavelmente  irá  reduzir  a  sua
produtividade. O ideal é executar testes unitários, que por sua própria natureza
são ordens de magnitude mais rápidas. A solução para o problema é usar {\it mock
  objects}.

\vspace{0.2cm}

Para ilustrar, tenha como base esta classe de domínio:

\lstset{language=Java}
\begin{lstlisting}[frame=single,            captionpos=b,            float=htbp,
    basicstyle=\fontsize{8}{9}\selectfont, showstringspaces=false] 
class Usuario {
  String nome
  String login
  static constraints = {
   nome(nullable:false, blank:false, maxSize:128, unique:true)
   login(nullable:false, blank:false, maxSize:16, unique:true)
  }
}
\end{lstlisting}

O teste unitário encontra-se na classe a seguir:

\lstset{language=Java}
\begin{lstlisting}[frame=single,            captionpos=b,            float=htbp,
    basicstyle=\fontsize{8}{9}\selectfont, showstringspaces=false]
import grails.test.*
class UsuarioTests extends GrailsUnitTestCase {
  protected void setUp() { super.setUp() }
  protected void tearDown() { super.tearDown() }
  void testConstraints() {
    mockDomain Usuario
    def usuario = new Usuario()
    assertFalse usuario.validate() // usuario nao validado pois
                                   // nao incluiu campos obrigatorios
    def usuarioOK = new Usuario(nome:"Maria", login: "maria")
    assertTrue usuarioOK.validate()
  }
}
\end{lstlisting}

Vale a pena salientar o seguinte  ponto: mesmo se tratando de um teste unitário,
o teste exercita  métodos que só existem  em tempo de execução: no  caso, o {\bf
  validate}.  Para  isto, foi  utilizado o método  {\bf mockDomain},  herdado de
{\bf GrailsUnitTestCase}. Este injeta na  classe de domínio todos os métodos que
uma classe  deste tipo deve ter –  como exemplos, os métodos  de validação, {\bf
  save()}, {\bf delete()}, etc. Assim é possível testar facilmente a validação.

No caso  de testes  de integração,  obviamente você não  precisa do  método {\bf
  mockDomain}, pois as classes já estão prontas.

\vspace{0.5cm}
\noindent {\bf Testes unitários  para controladores.} É muito comum encontrarmos
projetos nos  quais apenas  classes de domínio  são testadas.  Para  testar seus
controladores, você deve criar um  teste tal como faria normalmente. A diferença
é que  este teste não  estenderá a classe  {\bf GrailsUnitTestCase}, e  sim {\bf
  ControllerUnitTestCase}. 

\vspace{0.5cm}
\noindent {\bf  Executando os  testes.} Com a  aplicação funcionando,  o próximo
passo é  executar seus testes.   Para isto, deve-se  usar o comando  {\bf grails
  test-app}, que executará todos os seus testes.
\index{Comando!grails test-app} 

Para executar  apenas alguns  testes, basta passar  como parâmetro os  nomes dos
testes  excluindo  o sufixo  {\bf  Tests},  como  exemplo {\bf  grails  test-app
  Usuario}.

Para executar apenas testes unitários, execute {\bf grails test-app unit} e para
executar  apenas   os  testes  de  integração,  execute   {\bf  grails  test-app
  integration}.

Executados os seus testes, será  criado o diretório {\bf target/test-reports} em
seu projeto, contendo o relatório de execução dos testes.

\section{Estudos complementares}

\vspace{0.5cm}

Para  estudos complementares  sobre o  framework Grails  que foi  abordado nesse
material, o leitor interessado pode consultar as seguintes referências:

\begin{itemize}

\vspace{0.5cm}

\item  {BROWN, J.S.; ROCHER,  G. \emph{The  Definitive Guide  to Grails  2}. New
  York, USA: Apress, 2013.} 

\vspace{0.5cm}

\item  {SMITH, G.;  LEDBROOK, P.  \emph{Grails in  Action}. Greenwich,  CT, USA:
  Manning Publications Co., 2009.} 

\vspace{0.5cm}

\item {JUDD, C.M.; NUSAIRAT, J.F.;  SHINGLER, J. \emph{Groovy and Grails -- From
    Novice to Professional}. New York, USA: Apress, 2008.}  

\vspace{0.5cm}

\item  {K{\"O}ENIG, D.  et  al.  \emph{Groovy in  Action}.  Greenwich, CT,  USA:
  Manning Publications Co., 2007.}

\vspace{0.5cm}

\item  {VISHAL, L.;  JUDD,  C.M; NUSAIRAT,  J.F.;  SHINGLER, J.  \emph{Beginning
    Groovy, Grails and Griffon}. New York, USA: Apress, 2013.}

\end{itemize}
