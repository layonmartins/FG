\chapter{Considerações finais}\label{conclusoes}

Este  material  apresentou as  principais  funcionalidades  do framework  Grails
através da  descrição dos aspectos relacionados ao  desenvolvimento da aplicação
{\bf ControleBancario}.

\vspace{0.2cm}
\noindent Para finalizar esse material, esse capítulo apresenta o {\it overview}
de mais algumas funcionalidades presentes  em Grails que não foram abordadas nos
capítulos anteriores.

\section{Serviços Web}\index{Serviços Web}

\vspace{0.5cm}

Serviços web consistem  em fornecer uma API  web para a sua aplicação  web e são
geralmente implementados utilizando REST.

\subsection{REST}\index{Serviços Web!REST}

\vspace{0.5cm}

REST  não é  uma tecnologia  em si,  pode ser  considerada mais  como  um padrão
arquitetural. REST é muito simples e envolve apenas o uso simples de XML ou JSON
como meio de  comunicação, em conjunto com padrões na definição  de URLs que são
``representações'' do sistema subjacente, e  métodos HTTP, como GET, PUT, POST e
DELETE.

\vspace{0.2cm}

Cada método HTTP é  mapeado para um tipo de ação do  controlador. Por exemplo, o
método GET é mapeado para operações  de recuperação, o método PUT é mapeado para
operações de criação, o método POST é mapeado para operações de atualização e
assim por diante. Neste sentido REST  se encaixa muito bem com as operações CRUD
({\it Create}, {\it Read}, {\it Update} e {\it Delete}).

\vspace{0.2cm}

A abordagem mais fácil  para criar uma API REST em Grails  é expor uma classe de
domínio  como  um recurso  REST.   Isto  pode ser  feito  através  da adição  da
transformação   {\bf  grails.rest.Resource}   a  qualquer   classe   de  domínio
(Código~\ref{codREST}). 

\vspace{0.2cm}

Ao adicionar a transformação {\bf grails.rest.Resource} e especificar uma URI, a
classe de  domínio estará  automaticamente disponível como  um recurso  REST nos
formatos  XML ou  JSON.  Além  disso, a  transformação  registrará o  mapeamento
necessário   e    criará   um   controlador   (no   caso    do   exemplo,   {\bf
  LivroController}). Por  fim, é possível  alterar o padrão de  retorno (formato
JSON  ao  invés  do  formato  XML)  ao  definir  os  atributos  de  formatos  na
transformação {\bf grails.rest.Resource}.

\begin{lstlisting}[caption=Classe de domínio -- recurso REST, frame = trBL,
    float=htbp, label=codREST] 

import grails.rest.*

@Resource(uri='/livros', formats=['json', 'xml'])
class Livro {

    String titulo

    static constraints = {
        titulo blank:false
    }
}
\end{lstlisting}

Dessa forma, quando acessando a URL \url{http://localhost:8080/livros}, definida
no  atributo  {\bf  uri}   da  transformação  {\bf  grails.rest.Resource},  iria
renderizar uma resposta  para o acesso ao  recurso REST -- a lista  de livros no
formato JSON: 

\vspace{0.5cm}

\begin{cBox}
\begin{scriptsize}
\begin{verbatim}
[{"id":1,"titulo":"The Definitive Guide to Grails"},{"id":2,"titulo":"Grails in Action"}]
\end{verbatim}
\end{scriptsize}
\end{cBox}

\vspace{0.5cm}

Caso a  URL fosse \url{http://localhost:8080/livros.xml}, a resposta  seria -- a
lista de livros no formato XML:

\vspace{0.5cm}

\begin{cBox}
\begin{scriptsize}
\begin{verbatim}
<list>
  <livro id="1">
    <titulo>The Definitive Guide to Grails</titulo>
  </livro>
  <livro id="2">
    <titulo>Grails in Action</titulo>
  </livro>
</list>
\end{verbatim}
\end{scriptsize}
\end{cBox}

\vspace{0.5cm}

Para  maiores  informações  sobre   serviços  web  REST,  consulte  o  endereço:
{\footnotesize\url{https://grails.github.io/grails-doc/latest/guide/webServices.html}}.

\section{Automação de testes}

\vspace{0.5cm}

Essa seção apresenta algumas características de Grails que facilitam a automação
de  testes. Relembrando:  testes automatizados  te dão  segurança no  momento da
manutenção. No entanto, antes do início da discussão sobre tais características,
é fundamental relembrar  a diferença entre testes unitários  e de integração (ou
integrados). Para melhor explicar esses dois conceitos, utilizaremos a definição
presente no livro de Weissmann~\cite{wei15}: 

\vspace{0.2cm}

\begin{cBox}
{\it Grails nos  fornece por padrão dois tipos de  testes: unitário e integrado.
  O objetivo do teste unitário é  nos fornecer o ferramental necessário para que
  possamos verificar o  funcionamento de uma funcionalidade de  modo isolado, ou
  seja, sem que precisemos iniciar todo o sistema para isto.} 

\vspace{0.2cm}

{\it Isso nos  permite focar apenas na lógica  que estamos implementando naquele
  trecho do  sistema, mas por outro  lado, dado que  sempre precisamos interagir
  com outros  componentes arquiteturais, por exemplo, a  camada de persistência,
  um novo  problema surge. Como fazê-lo?  A resposta é  simples: criando versões
  falsas dos mesmos (os mocks). (...)}

\vspace{0.2cm}

{\it Testes unitários  nos fornecem, portanto, dependências do  nosso código que
  funcionam  em ``condições  ideais de  temperatura e  pressão'', ou  seja, elas
  funcionam exatamente  como gostaríamos,  o que pode  ser um problema,  mas nos
  ajuda a focar melhor naquilo que desejamos verificar.}

\vspace{0.2cm}
 
{\it Já testes integrados resolvem o problema da ``condição ideal de temperatura
  e pressão'' ao nos disponibilizar o sistema completo. Você não precisa mais de
  mocks, pois estará  usando o próprio sistema para  isso. São fundamentais para
  que vejamos  qual será  o comportamento do  sistema quando for  para produção,
  fornecendo-nos um ambiente que é o mais similar possível a este.}

\vspace{0.2cm}

\noindent ({WEISSMANN, H.~L. \emph{Falando  de Grails -- Altíssima produtividade
    no desenvolvimento web}. São Paulo: Casa do Código, 2015.)} 
\end{cBox}

\vspace{0.5cm}

Todos   os  testes   se  encontram   no  diretório   {\bf   src/test/groovy  (ou
  src/integration-test/groovy)} presente na raiz do projeto Grails. Toda vez que
é  criada  uma classe  de  domínio,  controlador  (e outros  artefatos),  testes
automaticamente são incluídos no diretório em desses diretórios.  Da mesma forma
que o  GORM é fortemente  baseado no framework  {\it hibernate}, o  arcabouço de
testes    presente     em    Grails    é    fortemente     baseado    no    {\it
  Spock}\footnote{~\url{https://code.google.com/archive/p/spock}}.  

\vspace{0.5cm}

Há três maneiras de se criar estes testes: (1) O Grails os cria automaticamente;
(2) A classe de teste é criada manualmente pelos desenvolvedores; e (3) A classe
de teste é criada usando o comando {\bf grails create-unit-test}.
\index{Comando!grails create-unit-test}

\vspace{0.5cm}

Assim  como diversos aspectos  do Grails,  aqui é  necessário ater-se  à algumas
convenções.  Toda classe de teste possui  o sufixo {\bf Spec} em seu nome. Sendo
assim, os testes unitários para a classe de domínio {\bf Usuario}}, por exemplo,
  ficariam em {\bf src/test/groovy/UsuarioSpec.groovy}. 

\vspace{0.5cm}

O comando {\bf grails  create-unit-test} ou {\bf grails create-integration-test}
deve receber  o nome do  teste unitário  ou de integração  a ser gerado.   Não é
necessário  incluir  o  {\bf  Spec}   no  final  do  arquivo,  Grails  o  inclui
automaticamente.  
\index{Comando!grails create-integration-test}

\subsection{Grails: TDD e BDD}

\vspace{0.5cm}

Grails prega a  automação dos testes unitários e  de integração, conscientizando
os desenvolvedores de  sua importância e dá suporte pleno  à adoção das técnicas
de teste de software {\it Test-Driven Development (TDD)} e {\it Behaviour-Driven
  Development (BDD)}. 

\vspace{0.2cm}

{\it   Test-Driven   Development}   faz    parte   dos   princípios   ágeis   de
desenvolvimento. Tal técnica foi criada por Kent Beck que impulsionou a ideia de
escrever testes automatizados antes da implementação do código~\cite{Beck12}.

\vspace{0.2cm}

A prática do {\it Test Driven  Development (TDD)} pode ser resumida de uma forma
bastante rudimentar  em uma frase:  escreva seus testes  antes do código.   É na
realidade  a aplicação  de um  modelo de  desenvolvimento baseado  em  rápidos e
pequenos ciclos~\cite{wei15}.  

\vspace{0.2cm}

{\it Test Driven Development (TDD)}  é um processo de desenvolvimento iterativo,
no  qual cada  iteração é  feita com  a escrita  de um  teste que  faz  parte da
especificação do  que será implementado.  Todas as  iterações são caracterizadas
por não  serem longas e propiciarem um  rápido feedback sobre o  código que esta
sendo  desenvolvido.  O  início  do   desenvolvimento  com  {\it  TDD}  é  feito
definindo-se uma  meta, ao contrário de  definir o código  que será implementado
para resolver determinado problema~\cite{joha12}. 

\vspace{0.2cm}

{\it Behaviour-Driven  Development (BDD)}  é uma técnica  ágil~\cite{North06}, e
uma evolução do  {\it Test-Driven Development (TDD)}, como  uma junção de outras
técnicas ágeis existentes que são úteis para o desenvolvimento de software, cujo
destaque para  sua utilidade se  deve à redução  dos custos com  modificações no
software  e  funções  do comportamento.   O  {\it  BDD}  como uma  técnica  ágil
incentiva  a   participação  e  a  colaboração   de  todos  os   membros  de  um
projeto~\cite{LMP10}. Ao  longo dos anos,  {\it BDD} progrediu para  um processo
que abrange análise  de requisitos e desenvolvimento do  código.  

\vspace{0.2cm}

Tal prática  de teste faz uso da  linguagem narrativa em junção  com a linguagem
ubíqua,  para  a  escrita  de  casos   de  testes  e  favorece  a  definição  do
comportamento  do sistema,  a linguagem  que  se aplica  ao BDD  é retirada  dos
cenários  criados durante  a fase  de análise  ou levantamento  de  requisitos e
permite uma comunicação entre todos os membros da equipe.


\subsection{BDD: um exemplo prático}

\vspace{0.5cm}

Nessa  seção apresentaremos  um  exemplo  simples da  utilização  do {\it  BDD},
através da ferramenta de testes {\it Spock}, na automação de testes. 

\vspace{0.2cm}

Ao adotarmos  o {\it BDD},  a nomenclatura dos  termos devem ser  enfatizados --
serão adotados os termos ``comportamento'' ({\it behaviour}) e ``especificação''
ao invés  de testes.  Para maiores  detalhes sobre o  porquê dessa nomenclatura,
consulte o artigo original sobre {\it Behaviour-Driven Development (BDD)} de Dan
North~\cite{North06}.  

\vspace{0.2cm}

Dessa forma, no jargão da ferramenta  de testes Spock, não temos testes, mas sim
comportamentos  -- sendo  um conjunto  de comportamentos,  considerado  como uma
especificação (arquivo {\bf XSpec.groovy}).  

\vspace{0.2cm}

O {\it BDD} incentiva os desenvolvedores  a escreverem testes cujo nome não seja
uma função ``usual'' tal como {\bf testCriarEndereco}, mas sim frases que possam
ser  compreendidas,  além  de  pelos  desenvolvedores,  também  pela  equipe  de
negócio.  

\vspace{0.2cm}

Figura~\ref{EndBehaFig} apresenta um dos comportamentos presente no arquivo {\bf
  EnderecoControllerSpec.groovy}.  Esse comportamento  testa a funcionalidade de
criação de instâncias da classe de domínio {\bf Endereco} -- ação {\bf create()}
do controlador {\bf EnderecoController}.  

\vspace{0.2cm}

\begin{figure}[htbp]
\begin{mdframed}
\begin{footnotesize}
\begin{verbatim}
@TestFor(EnderecoController)
@Mock(Endereco)
class AgenciaControllerSpec extends Specification {

// Demais comportamentos dessa especificação

  void "Test the create action returns the correct model"() {
      when:"The create action is executed"
           controller.create()

      then:"The model is correctly created"
           model.endereco!= null
  }
}
\end{verbatim}
\end{footnotesize}
\end{mdframed}
\caption{Especificação de testes para o controlador EnderecoController}
\label{EndBehaFig}
\end{figure}

As palavras  reservadas {\bf  when:} e {\bf  then:} demarcam blocos  de iteração
usados pela  ferramentas de  testes Spock. O  primeiro, {\bf when:}  (quando), é
usado  para  definir  as  condições  em  cima das  quais  deve-se  verificar  um
comportamento. É normalmente onde estão  definidas as variáveis que serão usadas
na verificação. Conforme  descrito na Figura~\ref{EndBehaFig}, é o  bloco em que
será checado  como o código  se comportará ao  criar uma instância da  classe de
domínio {\bf Endereco}.  

\vspace{0.2cm}

O próximo bloco, {\bf then:} , demarca aquilo que deve-se verificar. Neste caso,
deve  ser  escrita  uma  expressão  booleana  por linha.  Voltando  ao  caso  da
Figura~\ref{EndBehaFig}, deseja-se se certificar de que a instância da classe de
domínio {\bf Endereco} foi criada e não é nula.

\vspace{0.2cm}

Por fim,  é importante mencionar  que a especificação  {\bf EnderecoController},
através da anotação  {\bf @Mock}, obtém um {\it mock} da  classe de domínio {\bf
  Endereco} --  uma versão do  GORM feita especificamente para  testes unitários
que funciona inteiramente em memória.

\subsubsection{Escrevendo testes unitários para controladores}

\vspace{0.5cm}

Nessa  seção, vamos abordar  um simples  controlador hipotético  denominado {\bf
  TestController} no qual será simulado  as operações mais usuais na codificação
desse tipo de controlador.

\vspace{0.3cm}

\noindent {\bf Renderização de texto:}  Vamos começar pela situação mais simples
possível: uma ação  que renderiza um texto simples. Para  tal, imagine que nosso
controlador  possua uma  ação chamada  {\bf  index} cuja  implementação vemos  a
seguir: 

\begin{mdframed}
\begin{footnotesize}
\begin{verbatim}
def index() {
   render "Olá Mundo !"
}
\end{verbatim}
\end{footnotesize}
\end{mdframed}

\vspace{0.2cm}

Essa funcionalidade poderia ser testada pelo comportamento abaixo:

\vspace{0.2cm}

\begin{mdframed}
\begin{footnotesize}
\begin{verbatim}
void "testando a ação index"() {
    when:
      controller.index()
    then:
      response.text == "Olá Mundo !"
}
\end{verbatim}
\end{footnotesize}
\end{mdframed}

\vspace{0.3cm}

\noindent {\bf  Renderização de  GSP:} Outra  ação comum de  um controlador  é a
renderização  de  uma página  GSP.   Novamente,  algo  bastante simples  de  ser
verificado. Imagine a ação a seguir: 

\vspace{0.2cm}

\begin{mdframed}
\begin{footnotesize}
\begin{verbatim}
def renderGSP() {
   render view: 'renderGSP', model: [titulo: "Grails in Action"]
}
\end{verbatim}
\end{footnotesize}
\end{mdframed}

\vspace{0.2cm}

Essa  funcionalidade  poderia ser  testada  pelo  comportamento  abaixo. Não  se
verifica  apenas se  a página  correta foi  usada, mas  também se  o modelo  é o
esperado. 

\vspace{0.2cm}

\begin{mdframed}
\begin{footnotesize}
\begin{verbatim}
void "testando a ação renderGSP"() {
    when:
      controller.renderGSP()
    then: "o modelo deve ter a chave titulo"
      model.titulo == "Grails in Action"
      view == "/test/renderGSP"
}
\end{verbatim}
\end{footnotesize}
\end{mdframed}

\newpage

\noindent {\bf Testando redirecionamento}: Outra  ação comum de um controlador é
redirecionar  para outra  ação ou  controlador. Analogamente,  é simples  de ser
verificado.  Imagine a ação a seguir:

\vspace{0.2cm}

\begin{mdframed}
\begin{footnotesize}
\begin{verbatim}
def redirect(String papel){
   if (papel == "ROLE_ADMIN")
       redirect(controller:"admin")
   else
       redirect(controller: "cliente")
}
\end{verbatim}
\end{footnotesize}
\end{mdframed}

\vspace{0.2cm}

Essa  funcionalidade  poderia ser  testada  pelos  2  comportamentos abaixo  que
invocam a ação {\bf redirect} do controlador passando diferentes parâmetros. 

\vspace{0.2cm}

\begin{mdframed}
\begin{footnotesize}
\begin{verbatim}
void "testando a ação redirect - papel ROLE_ADMIN"() {
    when:
      controller.redirect("ROLE_ADMIN")
    then:
      response.redirectedUrl == "/admin"
}

void "testando a ação redirect - papel ROLE_CLIENTE"() {
    when:
      controller.redirect("ROLE_CLIENTE")
    then:
      response.redirectedUrl == "/cliente"
}
\end{verbatim}
\end{footnotesize}
\end{mdframed}

\section{Estudos complementares}

\vspace{0.5cm}

Para  estudos complementares  sobre o  framework Grails  que foi  abordado nesse
material, o leitor interessado pode consultar as seguintes referências:

\begin{itemize}

\vspace{0.5cm}

\item  {BROWN, J.S.; ROCHER,  G. \emph{The  Definitive Guide  to Grails  2}. New
  York, USA: Apress, 2013.} 

\vspace{0.5cm}

\item  {SMITH, G.;  LEDBROOK, P.  \emph{Grails in  Action}. Greenwich,  CT, USA:
  Manning Publications Co., 2009.} 

\vspace{0.5cm}

\item {JUDD, C.M.; NUSAIRAT, J.F.;  SHINGLER, J. \emph{Groovy and Grails -- From
    Novice to Professional}. New York, USA: Apress, 2008.}  

\vspace{0.5cm}

\item  {K{\"O}ENIG, D.  et  al.  \emph{Groovy in  Action}.  Greenwich, CT,  USA:
  Manning Publications Co., 2007.}

\vspace{0.5cm}

\item  {VISHAL, L.;  JUDD,  C.M; NUSAIRAT,  J.F.;  SHINGLER, J.  \emph{Beginning
    Groovy, Grails and Griffon}. New York, USA: Apress, 2013.}

\end{itemize}
